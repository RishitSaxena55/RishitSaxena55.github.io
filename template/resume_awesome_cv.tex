%!TEX TS-program = xelatex
%!TEX encoding = UTF-8 Unicode
%-------------------------
% Awesome CV Template
% Author: Rishit Saxena
% Based on: https://github.com/posquit0/Awesome-CV
%-------------------------

\documentclass[11pt, a4paper]{awesome-cv}

% Configure page margins with geometry
\geometry{left=1.4cm, top=.8cm, right=1.4cm, bottom=1.8cm, footskip=.5cm}

% Color for highlights
\colorlet{awesome}{awesome-red}

% Set false if you don't want to highlight section with awesome color
\setbool{acvSectionColorHighlight}{true}

% If you would like to change the social information separator from a pipe (|) to something else
\renewcommand{\acvHeaderSocialSep}{\quad\textbar\quad}

%-------------------------------------------------------------------------------
%	PERSONAL INFORMATION
%-------------------------------------------------------------------------------
\name{Rishit}{Saxena}
\position{ML Researcher{\enskip\cdotp\enskip}3D Computer Vision{\enskip\cdotp\enskip}Generative AI}
\address{Indian Institute of Information Technology Guwahati, India}

\mobile{(+91) 7082968644}
\email{rishitsaxena55@gmail.com}
\homepage{rishitsaxena55.github.io}
\github{RishitSaxena55}
\linkedin{rishit-saxena-12922531b}
\extrainfo{\href{https://x.com/SaxenaRishit55}{x.com/SaxenaRishit55}}

%-------------------------------------------------------------------------------
\begin{document}

% Print the header with above personal information
\makecvheader[C]

% Print the footer with 3 arguments(<left>, <center>, <right>)
\makecvfooter
  {\today}
  {Rishit Saxena~~~·~~~Resume}
  {\thepage}

%-------------------------------------------------------------------------------
%	CV/RESUME CONTENT
%-------------------------------------------------------------------------------

%-------------------------------------------------------------------------------
%	SECTION: Education
%-------------------------------------------------------------------------------
\cvsection{Education}

\begin{cventries}

  \cventry
    {B.Tech in Computer Science \& Engineering}
    {Indian Institute of Information Technology Guwahati}
    {Guwahati, India}
    {Aug 2024 - Apr 2028}
    {
      \begin{cvitems}
        \item {CPI: 9.23/10.0}
      \end{cvitems}
    }

\end{cventries}

%-------------------------------------------------------------------------------
%	SECTION: Experience
%-------------------------------------------------------------------------------
\cvsection{Experience}

\begin{cventries}

  \cventry
    {Research Intern}
    {Indian Institute of Information Technology Guwahati}
    {Guwahati, India}
    {May 2025 - Present}
    {
      \begin{cvitems}
        \item {Researching \textbf{Early Exit Networks (EENs)} for efficient inference in resource-constrained robotic environments}
        \item {Developed \textbf{INT8 quantization pipelines} achieving 4.5ms CPU inference for real-time applications}
        \item {Working on model compression techniques for edge deployment of Generative AI systems}
      \end{cvitems}
    }

  \cventry
    {ML Club Coordinator}
    {Maverics, Indian Institute of Information Technology Guwahati}
    {Guwahati, India}
    {Aug 2025 - Present}
    {
      \begin{cvitems}
        \item {Leading AI/ML community with 50+ active members; organizing workshops on Deep Learning and Computer Vision}
        \item {Mentoring students on research projects in neural rendering, object detection, and efficient ML}
      \end{cvitems}
    }

\end{cventries}

%-------------------------------------------------------------------------------
%	SECTION: Projects
%-------------------------------------------------------------------------------
\cvsection{Research Projects}

\begin{cventries}

  \cventry
    {PyTorch, Differentiable Rendering, Spherical Harmonics, 3D Vision}
    {3D Gaussian Splatting}
    {\href{https://github.com/RishitSaxena55/gaussian-splatting-pytorch}{GitHub}}
    {}
    {
      \begin{cvitems}
        \item {Built complete differentiable Gaussian rasterizer from scratch achieving real-time novel view synthesis}
        \item {Implemented 3D-to-2D projection, Spherical Harmonics for view-dependent color, adaptive density control}
        \item {Developed Clone/Split/Prune strategies scaling from 50K to 300K+ Gaussians during training}
      \end{cvitems}
    }

  \cventry
    {PyTorch, Volume Rendering, Neural Rendering, Positional Encoding}
    {NeRF: Neural Radiance Fields}
    {\href{https://github.com/RishitSaxena55/NeRF-PyTorch-}{GitHub}}
    {}
    {
      \begin{cvitems}
        \item {Engineered differentiable volume rendering pipeline from scratch with positional encoding}
        \item {Implemented hierarchical sampling (coarse-to-fine) for computational efficiency}
        \item {Achieved photo-realistic 3D reconstruction with novel view synthesis capabilities}
      \end{cvitems}
    }

  \cventry
    {PyTorch, Deep Learning, Region Proposal Network, Object Detection}
    {Faster R-CNN: Object Detection}
    {\href{https://github.com/RishitSaxena55/Faster-R-CNN}{GitHub}}
    {}
    {
      \begin{cvitems}
        \item {Implemented complete Faster R-CNN architecture from scratch including Region Proposal Network}
        \item {Built RoI pooling layer and multi-task loss function; trained on Pascal VOC dataset}
      \end{cvitems}
    }

  \cventry
    {Python, Autograd, Transformers, Multi-Head Attention}
    {TorchiFy-v1: Deep Learning Framework}
    {\href{https://github.com/RishitSaxena55/TorchiFy-v1}{GitHub}}
    {}
    {
      \begin{cvitems}
        \item {Built DL framework from first principles with custom autograd engine and computational graph}
        \item {Implemented Multi-Head Attention and LayerNorm with custom backward passes; supports GPT/Whisper}
      \end{cvitems}
    }

  \cventry
    {ResNet, Efficient ML, Model Optimization, Adaptive Inference}
    {EENet - Early Exit Networks}
    {\href{https://github.com/RishitSaxena55/EENet}{GitHub}}
    {}
    {
      \begin{cvitems}
        \item {Built Early Exit Networks with ResNet backbone for resource-efficient inference}
        \item {Supports MNIST, CIFAR10, SVHN, Tiny-ImageNet, ImageNet with instance-dependent computation}
      \end{cvitems}
    }

\end{cventries}

%-------------------------------------------------------------------------------
%	SECTION: Skills
%-------------------------------------------------------------------------------
\cvsection{Technical Skills}

\begin{cvskills}

  \cvskill
    {Languages}
    {Python, C++, C, Java, SQL, CUDA}

  \cvskill
    {ML Frameworks}
    {PyTorch, TensorFlow, JAX, Scikit-learn, NumPy, Pandas, Weights \& Biases}

  \cvskill
    {3D Vision}
    {3D Gaussian Splatting, NeRF, Differentiable Rendering, Volume Rendering, SLAM, SfM}

  \cvskill
    {Generative AI}
    {Transformers, LLMs, RAG, LoRA/PEFT, LangChain, LangGraph, GANs, VAEs}

  \cvskill
    {Efficient ML}
    {Quantization (INT8), Pruning, Early Exit Networks, ONNX, TFLite, Edge Deployment}

\end{cvskills}

%-------------------------------------------------------------------------------
\end{document}
